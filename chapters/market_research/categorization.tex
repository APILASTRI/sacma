\section{Categorization}

The term "Medical App" is broad and neither the iTunes nor Google app stores offer any kind of sub categorization. In order to get a better overview of what sort of Medical Apps are available it was necessary to manually browse the gathered data and assign categories to the apps.

A database management tool was created using the python based Django Web Framework. Django provides many tools that makes constructing and interacting with databases very easy. The built-in backend administration tool can be configured to browse, edit, and filter data.

In order to quickly browse through and categorize over 700 apps. The Django backend admin was configured so apps could be categorized one after the other with a minimum of clicks or scrolling. The user was presented a list of uncategorized apps. The first one is clicked. The user is then presented with a page displaying the title and summary text of the app and a field from which a category can be selected. Once saved, the app is no longer presented on the list, the user can select the next app at the top of the list.

\subsection{Description of Categories}
\noindent
\begin{itemize}
\item \textbf{Community} - provides some soft of social networking service through which the user can share data with her family or with a network of people suffering from similar disorders.
\item \textbf{Fun / Entertainment} - These apps have to real medical purpose. They are for enjoyment only.
\item \textbf{Alert / First Response} - Apps that assist first responders or that help users alert first responders that help is needed.
\item \textbf{Health / Lifestyle} - Relaxation and meditation apps, or ovulation and fertility reminders
\item \textbf{Resource Finder} - Apps that locate resources in the vicinitly, nearby pharmacies or care providers.
\item \textbf{Reminder} - Apps with timer or calendar functionality that might remind a user of an appointment or manage medince consumption.
\item \textbf{Algorithmic / Diagnostic} - These are apps that provide some sort of diagnositc information based on data that as been gathered by sensors or entered by the user. Examples are seizure detection apps, or stroke severity evalutaion apps.
\item \textbf{Learning / Educational / Reference} - By far the largest category, this includes apps that provide reference information about diseases or education material like anatomy apps for example.
\item \textbf{Organisational} - Apps in this category might help a user or practitioner organise, share or track data and documents. Examples are apps that help users track the status of their blood pressure or blood sugar levels, create health diaries, or manage clinical data and images.
\end{itemize}