                \begin{longtable}[H]{  | >{\bfseries}l | l | l | l | l | }

                    \hline

                    Name
                    & \multicolumn{4}{l |}{AppController} \\ \hline

                    Description
                    & \multicolumn{4}{p{8.5cm} |}{The AppController manages the global state of the app. It can disable UI elements when the app needs to wait for longer processes. It can restore the previous state of the app after it has been reopened or updated.} \\ \hline

                    Type
                    & \multicolumn{4}{l |}{Class}
                    \\ \hline

                    Implemented In
                    & \multicolumn{4}{l |}{app.js}
                    \\ \hline

                    Attributes
                    & Name & Type & \multicolumn{2}{p{6.5cm} |}{Description} \\ \hline
                        & \$state & class & \multicolumn{2}{p{6.5cm} |}{
                        The \$state object is provided by the angular ui-router and is used to define pages in the app. It can be used to query or set the ui state. The AppController uses the \$state to automatically move the user from the camera view to the analysis view when the border data has become available, for example.
                        } \\ \hline
                        & MDAppState & class & \multicolumn{2}{p{6.5cm} |}{
                        The MDAppState stores data the the AppController might need to restore it's state after a restart.
                        } \\ \hline
                        & uiDisabled & boolean & \multicolumn{2}{p{6.5cm} |}{
                        This variable is used to trigger changes in the ui. Through 2-way binding the ui components in the view will become disabled or enabled respectivly.
                        } \\ \hline
                        & tabStates & object & \multicolumn{2}{p{6.5cm} |}{
                        This javascript object stored the disabled-state of the individual tab components in the ui. The AppController disabiles or enables specific tab elements based on the state of the app and the data currently available.
                        } \\ \hline


                    Methods
                    & Name & Parameter & Return Type & Description \\ \hline

                        &


                \end{longtable}