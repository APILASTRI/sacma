\section{Background}

Modern smart phones have opened new possibilities in all kinds of fields. For medical applications the combination of high resolution cameras, sensor devices, powerful CPUs and mobile networking capabilities could even be life saving. By combining input devices with powerful algorithms it should be possible to provide important diagnostic information that can inform it’s user of impeding risks. One particularly interesting medical application would be the detection of melanoma in a skin lesion using a mobile phone.

Melanoma is a type malignant skin tumour that develops from benign melanocytic nevi. Although less frequent than other skin cancer types, it causes more deaths \cite{cancer_gov_skin}, and it’s incidence is increasing dramatically, especially in the young white population \cite{S_ez_2013}. Early diagnosis and treatment is vital.

Several non-invasive techniques exist which dermatologist can employ to visually make a diagnosis. The most common are pattern analysis, the ABCD rule, the 7-point checklist and the Menzies method, among others \cite{marghoob2004atlas}. These methods generally use a combination of defined visual clues and indicators to assess the risk of a skin lesion. CAD ( Computer Aided Diagnosis ) systems exists that can assist a dermatologist, and even provide automatic diagnosis with an accuracy comparable to that of doctors trained in dermoscopic techniques \cite{Filho2015}.


\section{Project Description}

This bachelor project will investigate the following questions:

\noindent
\begin{itemize}
\item What smartphone based medical apps are available that can track and assess the risk of skin lesions?
\item What methods and algorithms can be employed by mobile phones to calculate the risk?
\item What is necessary to build a mobile app that can provide a risk assessment of skin cancer melanoma from captured images?

\end{itemize}

\section{Goals}

\noindent
\begin{enumerate}
\item Research the medical app market to gain an overview of areas of development, what apps are available, what trends are discernible. Of special interest are apps that use internal sensors ( e.g., camera ) and mathematical algorithms to detect skin cancer melanoma.

\item Gain familiarity with image processing and analysis algorithms. Describe the basic principles of these algorithms.

\item Research and compare available data and algorithms (e.g. in computer vision / machine learning). Define the relevant attributes that are evaluated for the analysis. This can also include changes in size over time. Optionally use machine learning techniques to try to improve the results of the analysis algorithms

\item Based on the research and comparison choose a specific algorithm and data with which to proceed. The algorithm must be able to

\begin{enumerate}
\item identify the skin lesion correctly,
\item compute the relevant attributes,
\item use these attributes to classify the skin lesions whether it is possibly cancerous or not
\end{enumerate}

\item Implement the algorithm as a prototype ( using Python for example ). Using images of a skin lesion as input, the algorithm will provide a risk assessment as output. Implement tests that can calculate the accuracy of the algorithm. Compare the implemented algorithm with at least one other algorithm.


\item Create a concept for a medical mobile app that can utilise the features of modern smartphones to provide a risk assessment of skin cancer melanoma based on captured images. This includes a full requirements analysis, i.e. use cases, functional / non-functional requirements, architecture.

\item Implement as proof of concept specific aspects of the medical mobile app.

\end{enumerate}

\section{Expected Results}

\noindent
\begin{enumerate}
\item Results of the medical app market research including graphical presentation of trends and statistics.

\item Documentation of the available data and algorithms including high level excursion into the theory behind image processing and analysis algorithms.

\item Comparison of available data and algorithms. Definition of relevant attributes.

\item Decision, which algorithm to use.

\item Documentation of the implementation of the algorithm, including tests, evaluation and comparison with at least one other algorithm.

\item Documentation of a concept for a medial app, including defined use cases, requirements, architecture and implemented design patterns.

\item  Proof of Concept: implementation of specific aspects of the medical mobile app.

\end{enumerate}