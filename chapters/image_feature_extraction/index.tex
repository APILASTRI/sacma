A digital image is a matrix of pixels that contain color information, typically comprised of the 3 color channels red, green, and blue. A person can look at an image and quickly make statements about it's content. For instance, someone might look at an image of a street scene and be able to easily say "This is an image of a street in a town, there is one automobile, three people and a building in this scene". A person would easily be able to draw outlines around the objects and be able to differentiate between areas in each object.

For a computer to be able to "make statements" about the contents of an image it must use algorithms to group together and differentiate between objects in an image. The grouping together and differentiating of areas in an image is known as \textit{segmentation}. The statements that a computer can make about the content of an image are usually of statistical nature and are refered to as \textit{feature extraction}. Before segmentation an image is goes through a \textit{preprocessing} stage in order to remove or reduce irrelevant information or noise.

There are no universal algorithms for preprocessing, segmentation, or feature extraction. The algorithms employed depend on the context of the images to be analyed and the type of information that one is looking for.


\section{Preprocessing}

When capturing an image using a dermatoscope the skin is illuminated equally over the skin area to be photographed. The dermatoscope captures hi-resolution images with low levels of noise. The amount of preprocessing required is limited to colour enhancement to achieve better segmentation results. Using a digital camera like those found in smartphones requires introduces other challenges \cite{auto_seg}.

Unequal illumination or shadows in the image can make it more difficult for the segmentation algorithm to precisely recognise the lesions border. Glare from too much illumination, noise introduced by the circuitry of the sensor in the digital camera or hairs on the skin can also make it difficult to differentiate between the lesion area and healthy skin.

The preprocessing stage attempts to reduce the interference cause by these factors.


\subsection{Equalization}

\begin{figure}[H]
    \includegraphics[width=\textwidth,keepaspectratio]{assets/image_processing/equalization/figure_01.png}
    \caption{Image Equalization Example A}
    \label{fig:eq_A}
\end{figure}
\begin{figure}[H]
    \includegraphics[width=\textwidth,keepaspectratio]{assets/image_processing/equalization/figure_02.png}
    \caption{Image Equalization Example B}
    \label{fig:eq_A}
\end{figure}
\begin{figure}[H]
    \includegraphics[width=\textwidth,keepaspectratio]{assets/image_processing/equalization/figure_03.png}
    \caption{Image Equalization Example C}
    \label{fig:eq_A}
\end{figure}

\subsection{Median Blur}
\subsection{Gaussian Blur}
\section{Segmentation}

\subsection{Iterative Thresholding}
\subsection{Region Selection}
\subsection{Super Pixels}

\input{chapters/image_feature_extraction/feature_extraction}

