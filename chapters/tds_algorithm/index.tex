\section{Description of the Algorithm}

Early detection of melanoma greatly increases the chances of successful treatment. A biopsy can be performed in order to gain a definitive diagnosis. However, biopsies are invasive, painful and take time. There are also visual markers Dermatologists look for in order to make a risk Assessment. The ABCD Rule, also known as Stokes or TDS Calculation, looks for 4 sets of features. Based on the features the Total Dermoscopy Score (TDS) is calculated.

The 4 sets of features are Asymmetry (A), Border Irregularity (B), Color (C) and Differential Structure or Diameter (D).

Asymmetry can have a value of 0, 1 or 2 depending on the symmetry of the lesion. Where 0 is symmetric and 2 is asymmetric. A value of 1 indicates at least one axis was found across which symmetry exists. The Border score is an integer value from 0 to 8 indicating the presence of border irregularities in 8 regions. Color is an integer value from 1 to 6 indicating the presence of one to six specific colors. Similarly, the value for D indicates the presence of one to five distinct structures or textures. Alternatively, in some literature\cite{Siddiq_2015} D is defined as Diameter, where a diameter greater than 6mm results in a value of 5, otherwise 1.

The final TDS Score is the weighted sum of the ABCD Values.

\begin{equation}
TDS = A * 1.3 + B * 0.1 + C * 0.5 + D * 0.5
\end{equation}

The final TDS score can have a value between 1.0 and 8.9 .

\begin{table}[H]
\small
    \begin{tabular}{ | l | p{3.5cm} | l | p{3.5cm} |}
    \hline
    Evaluation & TDS Score \\ \hline
    Benign & < 4.75  \\ \hline
    Suspicious & 4.75 to 5.45  \\ \hline
    Malignant & > 5.45  \\ \hline

    \end{tabular}

    \caption{TDS Evaluation}
    \label{fig:tds_eval}

\end{table}


\section{Automatic Calculation of ABCD Values}

The ABCD Rule is used by Dermatologists to differentiate benign from malignant melanocytic tumors. Because it is a clearly defined rule based on easily recognizable visual features it is easy to apply. Clinicians with limited dermoscopy experience achieve better results using the ABCD rule than other methods \cite{Weigert_2012}. The following sections detail methods to automate the calculation of the ABCD scores.

\subsection{Asymmetry}

\subsection{Border}

\subsection{Color}

\subsection{Differential}

\section{Two Examples}
\subsection{Postitiv Example}

\subsection{False Example}

ABCD Values were calculated but classification result was false.

\section{Results of Algorithm}

Performance Evaluation

Chapters 11.1, 11.2, maybe 11.3