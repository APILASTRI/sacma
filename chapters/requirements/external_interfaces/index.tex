\subsubsection{User Interfaces}

        \begin{itemize}[leftmargin=1.4cm]
            \item[UI-1 :] The UI shall guide the Smartphone User sequentially through the process of capturing and analysing an image. The Smartphone User should not be able to jump through states in wrong sequence.
            \item[UI-2 :] The UI shall allow the Smartphone User to capture an image easily, using only one hand.
            \item[UI-3 :] When waiting for a process to finish or a response from an online server the UI should visually indicate that it is waiting for a long process. ( i.e. throbber, spinning wheel, progress indicator )
            \item[UI-4 :] The UI shall allow the Smartphone User to select her or his preferred language ( i.e. localisation )
            \item[UI-5 :] The UI shall conform to native UI standards as much as possible.
            \item[UI-6 :] Unless in conflict with UI-5 the UI shall have the same components across multiple platforms.

        \end{itemize}

\subsubsection{Software Interfaces}

    \paragraph{Smartphone Camera }

        The MDApp will interface with the Smartphone's camera hardware via the systems camera API.

        \begin{itemize}[leftmargin=1.4cm]
            \item[SI-1.1 :] The MDApp shall activate the camera when needed and be able to receive a live preview stream from the camera.
            \item[SI-1.2 :] If the native API allows, the MDApp should set to camera’s focus mode to Macro.
            \item[SI-1.3 :] If the native API allows, the MDApp should register a call back to be informed when the camera’s autoFocus has focused.
            \item[SI-1.4 :] The MDApp shall signal the camera to capture the highest resolution image possible when the Smartphone User triggers an image capture event.

        \end{itemize}

    \paragraph{Smartphone File System}

        Captured images and Border Analysis images must be stored on the Smartphone File System in such a manner so that they are accessible by the Smartphone User in the native methods. The native file access, file handling, and backup methods for the respective Smartphone platform shall apply to files created and captured by the MDApp

        \begin{itemize}[leftmargin=1.4cm]
            \item[SI-2 :] The MDApp will save files to the Smartphone's native file system.

        \end{itemize}

    \paragraph{Smartphone Email Service}

            \begin{itemize}[leftmargin=1.4cm]
            \item[SI-3 :]

        \end{itemize}

\subsubsection{Communication Interfaces}

    \paragraph{Border Extraction Service }

        The Border Extraction Service is an online web based service that the MDApp can communicate with for the purpose of extracting the precise border information from a captured image of a skin lesion.

        \begin{itemize}[leftmargin=1.4cm]
            \item[CI-1.1 :] The MDApp will upload a captured skin lesion image to a web-based Border Extraction Service. The files will be uploaded as part of a multipart/form-data html POST method.
            \item[CI-1.2 :] The MDApp will receive a confirmation from the service as JSON, including an id that will be used to poll the service for progress.
            \item[CI-1.3 :] The MDApp will poll the online service regularly, the service will indicated that the Border Extraction processes is in progress or has finished, in which case it will also include the url where the border image can be downloaded.
            \item[CI-1.4 : ] The MDApp will download the border data image from the Border Extraction Service

        \end{itemize}

    \paragraph{Risk Assessment Service }

        The Risk Assessment Service is an online web based service that the MDApp can communicate with for the purpose of evaluating the risk of a skin lesion being a melanoma.

        \begin{itemize}[leftmargin=1.4cm]
            \item[CI-2.1 :] The MDApp will upload a captured skin lesion image and border image to a web-based Risk Assessment Service. The files will be uploaded as part of a multipart/form-data html POST method.
            \item[CI-2.2 :] The MDApp shall receive a confirmation from the service as JSON, including an id that will be used to poll the service for progress.
            \item[CI-2.3 :] The MDApp will poll the online service regularly, the service will indicated that the Risk Assessment processes is in progress or has finished, in which case it will respond with the results of the assessment as JSON.

        \end{itemize}