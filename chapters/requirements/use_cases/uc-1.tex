\begin{table}[H]
    \begin{tabular}[t]{ | >{\bfseries}l | p{9.5cm} |} \hline

    ID &
        UC-1 \\ \hline
    Name &
        Capture Image \\ \hline
    Priority &
        \\ \hline
    Description &
        As a user I can capture an image of the skin lesion that I would like to have analysed. \\ \hline
    Dependencies &
        None \\ \hline
    Trigger event &
        The user activated the application and selected the "image capture" navigation item. \\ \hline
    Preconditions &
         \\ \hline
    Postconditions &
        \\ \hline
    Result &
        The captured image is saved and the system initiates the border calculation\\ \hline
    Main scenario
    &

    \begin{description}[topsep=0cm,align=left]
        \item [1.]Point the camera at the skin lesion.
        \item [2.]Rotate and move the camera until the skin lesion is centered and optimally sized.
        \item [3.]The user touches the screen to capture the image.
        \item [4.]The user is notified ( beep ) that the image has been captured
        \item [5.]The user can repeat from the begining
    \end{description}

    \\ \hline

    Alternate scenarios &
        None \\ \hline

    Comments &
        It is important that a user can capture the image with just one hand. If a lesion is located on a user's hand or arm, it's not possible to use two hands. UI should account for this if possible. \\ \hline

    \end{tabular}
  \caption{Use Case 1}
  \label{fig:uc_1}
\end{table}