\begin{table}[H]
    \begin{tabular}[t]{ | >{\bfseries}l | p{9.5cm} |} \hline

    ID &
        UC-2 \\ \hline
    Name &
        Confirm calculation of skin lesion’s border \\ \hline
    Priority &
        \\ \hline
    Description &
        As a user I want to confirm that the skin lesion’s borders have been properly calculated. \\ \hline
    Dependencies &
        UC-1 \\ \hline
    Trigger event &
        The user selected the ”confirm border” navigation item. \\ \hline
    Preconditions &
        At least one image has been captured and border calculation could complete. \\ \hline
    Postconditions &
        \\ \hline
    Result &
        After positiv confirmation, the system initiates the risk assessment calculation for the image. If the image can not be positively confirmed, the user can recapture new images ( UC-1 ) \\ \hline
    Main scenario
    &

        \begin{description}[align=left]
            \item [1.]The user is presented with a list of images.
            \item [2.]The following step is repeated for each image in the list.
            \item [3.]The user confirms that the border of the lesion has been precisely calculated.
        \end{description}

    \\ \hline

    Alternate scenarios &
        \begin{description}[align=left]
            \item [A1.] Border calculation for image has not completed.
            \item [A1.3] The user can refresh the image preview until the results of the border are visible.
            \item [A1.4] The user confirms that the border of the lesion has been precisely calculated.
        \end{description}
        \begin{description}[align=left]
            \item [A2] Border calculation is not precise or has failed.
            \item [A2.3] The user confirms that the border of the lesion has not been precisely calculated.
            \item [A2.4] The image is deleted.
        \end{description}
    \\hline

    Comments &
        \\ \hline

    \end{tabular}
  \caption{Use Case 2}
  \label{fig:uc_2}
\end{table}