    
        \subsubsection{Logical Data Model}

        The entity-relationship diagram in figure \ref{fig:partial_data_model} is a high level view of the main groups of information and their interconnections. The figure uses the Peter Chen notation, where the diamonds represent the relationships between pairs of entities. Cardinality of the relationships is shown as numbers on the lines that connect the relationships to the entities. The entities are the boxes. The relationships  can be read for example as “Metadata describes the Lesion Image” (active voice) or “The Lesion Image is described by Metadata” (passive voice). The order of the entities in the diagram does not define the relationship, the object and subject must be inferred. For instance the relationship between Archive and Lesion Image is read “The Archive stores many Lesion Images” has the opposite positioning in the diagram compared with the relationship between Metadata and Lesion Image. The “direction” of the relationship is specified in the following data dictionary along with the entities attributes in detail.

            \begin{figure}[H]
                \centering
                \includegraphics[width=\textwidth]{assets/requirements/EntRel.pdf}
                \caption{Patial entity relationship model of the Melanoma Detection App}
                \label{fig:partial_data_model}
            \end{figure}


        \subsubsection{Data Dictionary}

        The Data Dictionary \ref{fig:partial_data_model} lists the data entities in detail. Entities can be primitives or composites. Primitives are simple data elements like an integer, float, or string. Composites are collections of entities which can themselves be primitives or composites. The entities that are part of a composite are listed in the Composition or Data Type column separated by a “+” sign. Each entity in a composite will also be listed separately and described in the data dictionary below. If applicable the length and range of possible values will be described for primitiv entities.

                {\small
                \begin{longtable}[H]{ | p{2.5cm} | p{2.5cm} | p{2.5cm} | p{1.0cm} | p{2.5cm} | }

                    \hline
                    \textbf{Data element} & \textbf{Description} & \textbf{Componsition or data type} & \textbf{Length} & \textbf{Values} \\ \hline

                    \hypertarget{lesion_image}{Lesion Image} & reference to the captured image &

                        \specialcell[t]{Image ID
                           \\ + File Path
                           \\ + Creation Date
                        }

                     & & \\ \hline

                    Image ID & unique identifier for an image &
                    integer & 11 & system-generated sequential integer beginning with 1 \\ \hline

                    File Path & location of a file on the file system &
                    string & 256 &  \\ \hline

                    Creation Date & date and time that a file was created &
                    DDMMYYYY:HHSS &  &  \\ \hline

                    Metadata & information describing the lesion and it's location &

                        \specialcell[t]{\hyperlink{lesion_image}{Lesion Image}
                           \\ + Lesion Description
                           \\ + Lesion Location
                        }

                     & & \\ \hline

                    Lesion Description & User defined text that describes the lesion &
                    alphanumeric &  &  \\ \hline

                    Lesion Location & information describing the lesion location on the body &

                        \specialcell[t]{\hyperlink{lesion_image}{Lesion Image}
                           \\ + Body Location ID
                           \\ + X Coordinate
                           \\ + Y Coordinate
                        }

                     & & \\ \hline


                    Body Location ID & id of a region on the body &
                    integer & 2 &

                        \specialcell[t]{
                            1 : Head Front \\
                            2 : Torso Front \\
                            3 : Right Upper Arm \\
                            4 : Left Upper Arm \\
                            5 : Right Forearm \\
                            6 : Left Forearm \\

                        }

                     \\ \hline

                    X Coordinate & x coordinate of the lesion in the region specified by the body location id &
                    integer & 128 &  \\ \hline

                    Y Coordinate & y coordinate of the lesion in the region specified by the body location id &
                    integer & 128 &  \\ \hline

                    Border Image & data that defines the outline of the lesion area &

                        \specialcell[t]{\hyperlink{lesion_image}{Lesion Image}
                           \\ + File Path
                           \\ + Creation Date
                        }

                     & & \\ \hline

                    Risk Assessment &  &

                        \specialcell[t]{\hyperlink{lesion_image}{Lesion Image}
                            \\ + Assessment Date
                            \\ + Version Number
                            \\ + SFA Major
                            \\ + SFA Minor
                            \\ + Border Irregularity
                            \\ + Color Score
                            \\ + \hyperlink{tds_score}{TDS Score}
                        }

                    & & \\ \hline

                    Assessment Date & date and time that the risk assessment was calculated &
                    DDMMYYYYHHSS &  &  \\ \hline

                    Version Number & version number of the risk assessment service software &
                    integer & 4 &  \\ \hline

                    SFA Major & measure of symmetry of the lesions major axis &
                    integer & 3 & 0 - 360 \\ \hline

                    SFA Minor & measure of symmetry of the lesions minor axis &
                    integer & 3 & 0 - 360 \\ \hline

                    Border Irregularity & measure of irregulatiry of the lesion's border &
                    integer & 1 & 0 - 8 \\ \hline

                    Color Score & number of specific colors found in the lesion's image &
                    integer & 1 & 0 - 8 \\ \hline

                    \hypertarget{tds_score}{TDS Score} & weighted linear combination of the image features summarizing the risk assessment  &
                    float &  & 1 - 8.9 \\ \hline

                    Archive &  &

                        \specialcell[t]{Archive ID
                            \\ + \hyperlink{lesion_image}{Lesion Image}
                            \\ + Metadata
                            \\ + Border
                            \\ + Risk Assessment
                            \\ + Archive Date
                        }

                     & & \\ \hline

                    Active Element & This points to the Lesion Image currently being processed and on which the graphical user interface is acting on. &

                        \specialcell[t]{\hyperlink{lesion_image}{Lesion Image}
                            \\ + Metadata
                            \\ + Border
                            \\ + Risk Assessment
                            \\ + Archive Date
                        }

                     & & \\ \hline
                    \caption{Partial data dictionary for the Melanoma Detection App}
                    \label{fig:data_dictionary}
                \end{longtable}}


        \subsubsection{Data analysis}
            \paragraph{CRUD Matrix}

                A CRUD ( Create, Read, Update, Delete ) matrix can highlight which use cases interact with which data entities and in which way. Every entity that is read or deleted also needs to be created somewhere. The CRUD matrix in figure \ref{fig:crud} shows that this is the case.

                \begin{figure}[H]
                    \centering
                    \includegraphics[width=\textwidth]{assets/requirements/CRUD.pdf}
                    \caption{CRUD matrix for Melanoma Detection App}
                    \label{fig:crud}
                \end{figure}

        \subsubsection{Reports}
            \paragraph{Email Report}
                When the Smartphone User chooses to send an archive as an email, the data must be flattened into a format that is email compatible. The body of the email will be an html document that displays the lesion image, the calculated border, the risk assessment data and the associated metadata.


