In section 2.2.1 categories of medical smart phone apps were defined. Apps in the same categories might have a similar feature set and capabilities. An app in the \textit{Alert / First Response} category might have a database that stores important contacts in the user’s area and a user interface that allows the user to quickly alert a contact in an emergency situation. An app in the \textit{Learning / Educational / Reference} category might be modelled as a quiz app which presents the user with a question and several possible answers to choose from. Answers from several questions will can be evaluated and a score is presented to the user at the end of a session. The app would be implemented with a database of questions and weighted answers, as well as an interface that presents the questions to the user sequentially.

One of the goals of this project is to define requirements for an app that provides a risk assessment of a lesion being benign or malignant. A high level description of the primary function of the app is the following:

\begin{itemize}[label={}]
\item The smart phone app allows the user to capture and analyse an image of a skin lesion and provide a risk assessment to the user of the lesion being a malignant melanoma.

\end{itemize}

The user might also like to save the image and results in oder to be able to compare it with other assessments in the future, or to review the assessment with a dermatologist. For comparison it might be useful to save associated metadata, such as the date and location on the body of the lesion. For convenience it might be nice to send the assessment and image via email to a specialist for review.
Secondary functionality can be described as follows:

\begin{itemize}[label={}]
\item The app allows the user to save or archive the image and corresponding assessment for future comparison and review.
\item The user can add, edit, and save metadata associated with the image of the lesion.
\item The user can browse archived images, assessments, and associated metadata.
\item The user can send a set of images with associated assessment and metadata via email.
\end{itemize}

From this high level description some basic assumptions about the app’s architecture can already be made. The app will require access to a database that can store information about an image, results of the analysis, and metadata. The app will require a user interface that allows a user to browse and edit data associated with an image. The app requires access to the smart phone’s camera api in order to capture images.
Another consideration that will impact design decisions is the investment already made in developing the image processing and analysis algorithms. Choosing python as the basis for the numerical algorithms and leveraging python based scientific computing libraries has implications because python code is not easily portable to iOS or Android devices. In order to efficiently utilise the algorithms as they are, and to continue to update and improve them, the image processing and analysis algorithms should be implemented as online services.

\begin{itemize}[label={}]
\item The image processing and risk assessment algorithms will be implemented as online services.
\end{itemize}

In order to formally elicit the requirements the hight level description will be broken down into structured use cases. A set of requirements will be extracted from the use cases.

\section{Use Cases}
        \begin{table}[H]
            \centering
            \includegraphics[width=\textwidth]{assets/requirements/uc/usecase_01.pdf}
            \caption{UC-1}
            \label{fig:uc-1}
        \end{table}
        \begin{table}[H]
            \centering
            \includegraphics[width=\textwidth]{assets/requirements/uc/usecase_02.pdf}
            \caption{UC-2}
            \label{fig:uc-2}
        \end{table}
        \begin{table}[H]
            \centering
            \includegraphics[width=\textwidth]{assets/requirements/uc/usecase_03.pdf}
            \caption{UC-3}
            \label{fig:uc-3}
        \end{table}
        \begin{table}[H]
            \centering
            \includegraphics[width=\textwidth]{assets/requirements/uc/usecase_04.pdf}
            \caption{UC-4}
            \label{fig:uc-4}
        \end{table}
        \begin{table}[H]
            \centering
            \includegraphics[width=\textwidth]{assets/requirements/uc/usecase_05.pdf}
            \caption{UC-5}
            \label{fig:uc-5}
        \end{table}
        \begin{table}[H]
            \centering
            \includegraphics[width=\textwidth]{assets/requirements/uc/usecase_06.pdf}
            \caption{UC-6}
            \label{fig:uc-6}
        \end{table}
        \begin{table}[H]
            \centering
            \includegraphics[width=\textwidth]{assets/requirements/uc/usecase_07.pdf}
            \caption{UC-7}
            \label{fig:uc-7}
        \end{table}
        \begin{table}[H]
            \centering
            \includegraphics[width=\textwidth]{assets/requirements/uc/usecase_08.pdf}
            \caption{UC-8}
            \label{fig:uc-8}
        \end{table}
        \begin{table}[H]
            \centering
            \includegraphics[width=\textwidth]{assets/requirements/uc/usecase_09.pdf}
            \caption{UC-9}
            \label{fig:uc-9}
        \end{table}
        \begin{table}[H]
            \centering
            \includegraphics[width=\textwidth]{assets/requirements/uc/usecase_10.pdf}
            \caption{UC-10}
            \label{fig:uc-10}
        \end{table}


\section{Prioritisation}
\subsubsection{User Interfaces}

            This section describes the components of the user interface. The components are grouped into views which allow the user to interact with a particular state or feature of the application. Some views may extend beyond the edges of the smart phone's screen size, in that case the figure will attempt to show multiple versions of a view. The figures shown do not represent every possible combination of states the application may have.

            \paragraph{UI-1: Home Screen}

            The Home Screen is the view that will appear to the user when the MDApp is first started. It will act as a jump off point to guide the user through the app, showing quick links with a short description. When the MDApp is started for the first time, there will be no data or images available to view in the Analysis or Archive views. The MDApp will therefore not display the quick link for the Analysis View or the Archive View, only the Camera View.

                \begin{figure}[H]
                    \centering
                    \includegraphics[height=7cm]{assets/GUI/HOME_01.pdf}
                    \caption{UI-1: Home Screen}
                    \label{fig:ui-1}
                \end{figure}

                \begin{itemize}
                    \item[1] View Title, every view shall have a title.
                    \item[2] Quick link to the Camera View.
                    \item[3] Quick link to the Analysis View, only visible if an image has already been captured.
                    \item[4] Quick link to the Archive View, only visible if images have been saved to the archive.
                    \item[5] Tab-bar navigation element. This component is always visible, in all of the app’s views.
                \end{itemize}

            \paragraph{UI-2: Camera View}

            The Camera View will automatically start the real time camera preview, displaying an image of what the camera is currently pointing at. Some visual indicators will help the user to center the skin lesion in the image. Because the user might be taking an image of her or his own body it might be difficult to click a button while holding the camera steady. Therefore touching anywhere on the camera preview area will trigger an image capture. An audible “shutter” noise will also signal the user that an image has been captured.

                \begin{figure}[H]
                    \centering
                    \includegraphics[height=7cm]{assets/GUI/CAMERA_01.pdf}
                    \caption{UI-2: Camera View}
                    \label{fig:ui-2}
                \end{figure}

                \begin{itemize}
                    \item[1] View Title, every view will have a title.
                    \item[2] Realtime camera preview.
                    \item[3] Crosshairs, visual markers that indicate where the center of the image is. The user should attempt to keep the lesion within the markers, but filling the marked area as much as possible. (limited by the focal range of the smart phone’s camera)

                \end{itemize}

            \paragraph{UI-3: Analysis View}

            The Analysis View presents the current state of processing and the results of any finished calculations. When the border calculation is finished, the Smartphone User is promoted to confirm that the border was precisely calculated. If not the app will return the Smartphone User to the Camera View to capture another version of the image. The risk assessment data will be presented to the Smartphone User below the border image along with a summary and recommendation as text ( not shown in figure ref[] ). The Smartphone User can also add addition text as metadata and specify where on her/his body the skin lesion was located. The Smartphone User also has the option to save the image and data to the archive.

                \begin{figure}[H]
                    \centering
                    \includegraphics[height=7cm]{assets/GUI/ANALYSIS.pdf}
                    \caption{UI-3: Analysis View}
                    \label{fig:ui-3}
                \end{figure}

                \begin{itemize}
                    \item[1] Save button.
                    \item[2] Border Analysis preview
                    \item[3] Confirm and Cancel buttons with which the Smartphone User may confirm or reject that the border has been precisely calculated.

                    \item[4] Once confirmed, the button becomes an indicator that the border was confirmed. It is no longer clickable in this state.

                    \item[5] Additional metadata may be entered by the Smartphone User as text. Upon clicking a text input field the smartphone’s native text entry keyboard element will appear.

                \end{itemize}

            \paragraph{UI-4: Archive List View}

                \begin{figure}[H]
                    \centering
                    \includegraphics[height=7cm]{assets/GUI/ARCHIVE_01.pdf}
                    \caption{UI-4: Archive List View}
                    \label{fig:ui-4}
                \end{figure}

            \paragraph{UI-5: Archive Detail View}

                \begin{figure}[H]
                    \centering
                    \includegraphics[height=7cm]{assets/GUI/ARCHIVE_02.pdf}
                    \caption{UI-5: Archive Detail View}
                    \label{fig:ui-5}
                \end{figure}


        \subsubsection{Software Interfaces}

            \paragraph{SI-1 : Smartphone Camera }
                The MDApp will interface with the Smartphone's camera hardware via the systems camera API.

                \begin{itemize}[leftmargin=1.4cm]
                    \item[SI-1.1 : ] The MDApp shall activate the camera when needed and be able to receive a live preview stream from the camera.
                    \item[SI-1.2 : ] If the native API allows, the MDApp should set to camera’s focus mode to Macro.
                    \item[SI-1.3 : ] If the native API allows, the MDApp should register a call back to be informed when the camera’s autoFocus has focused.
                    \item[SI-1.4 : ] The MDApp shall signal the camera to capture the highest resolution image possible when the Smartphone User triggers an image capture event.

                \end{itemize}

            \paragraph{SI-2 : Smartphone File System }
                Captured images and Border Analysis images must be stored on the Smartphone File System in such a manner so that they are accessible by the Smartphone User in the native methods. The native file access, file handling, and backup methods for the respective Smartphone platform shall apply to files created and captured by the MDApp

        \subsubsection{Communication Interfaces}

            \paragraph{CI-1 : Border Extraction Service }

                The Border Extraction Service is an online web based service that the MDApp can communicate with for the purpose of extracting the precise border information from a captured image of a skin lesion.

                \begin{itemize}[leftmargin=1.4cm]
                    \item[CI-1.1 : ] The MDApp will upload a captured skin lesion image to a web-based Border Extraction Service. The method for uploaded as part of a multipart/form-data html POST method.
                    \item[CI-1.2 : ] The MDApp will receive a confirmation from the service as JSON, including an id that will be used to poll the service for progress.
                    \item[CI-1.3 : ] The MDApp will poll the online service regularly, the service will indicated that the Border Extraction processes is in progress or has finished, in which case it will also include the url where the border image can be downloaded.
                    \item[CI-1.4 : ] The MDApp will download the border data image from the Border Extraction Service

                \end{itemize}

            \paragraph{CI-2 : Risk Assessment Service }

                The Risk Assessment Service is an online web based service that the MDApp can communicate with for the purpose of evaluating the risk of a skin lesion being a melanoma.

                \begin{itemize}[leftmargin=1.4cm]
                    \item[CI-2.1 : ] The MDApp will upload a captured skin lesion image to a web-based Border Extraction Service. The method for uploaded as part of a multipart/form-data html POST method.

                \end{itemize}

\section{Requirements}
\section{Prioritized Requirements}

\begin{longtable}{ | l | l | l | l | l | l | l |}
\hline
ID & Priority & Name & Implemented  \\ \hline

REQ-F-1 & - & Preview Camera Input & -  \\ \hline
REQ-F-2 & - & Capture Camera Input & -  \\ \hline
REQ-F-3 & - & Border Extraction & -  \\ \hline
REQ-F-4 & - & Browse Images and Results & -  \\ \hline
REQ-F-5 & - & Select Image To View in Detail & -  \\ \hline
REQ-F-6 & - & Border Confirmation & -  \\ \hline
REQ-F-7 & - & Feature Extraction & -  \\ \hline
REQ-F-8 & - & Feature Extraction & -  \\ \hline


\caption{Prioritzed Requirement List}
\label{fig:prio_req}
\end{longtable}


\begin{table}[H]
    \begin{tabular}[t]{ | >{\bfseries}l | p{9.5cm} |}

    \hline
    ID
    &  REQ-F-1 \\ \hline

    Name
    & Preview Camera Input \\ \hline

    Description
    &  The system will let the user view a preview of the camera's input in realtime \\ \hline

    Preconditions
    & None \\ \hline

    Acceptance Tests
    & \\ \hline

    Relations
    & UC-1 \\ \hline

    Comments
    &  \\ \hline

    \end{tabular}

    \caption{Functional Requirement 1}
    \label{fig:req_f_1}

\end{table}
\begin{table}[H]
    \begin{tabular}[t]{ | >{\bfseries}l | p{9.5cm} |}

    \hline
    ID
    &  REQ-F-1 \\ \hline

    Name
    & Capture Camera Input \\ \hline

    Description
    &  The system will let the user capture an image from the camera's input. \\ \hline

    Preconditions
    & None \\ \hline

    Acceptance Tests
    & \\ \hline

    Relations
    & UC-1 \\ \hline

    Comments
    &  \\ \hline

    \end{tabular}

    \caption{Functional Requirement 1}
    \label{fig:req_f_1}

\end{table}
\begin{table}[H]
    \begin{tabular}[t]{ | >{\bfseries}l | p{9.5cm} |}

    \hline
    ID
    &  REQ-F-3 \\ \hline

    Name
    & Border Extraction \\ \hline

    Description
    &  The system must be able to calculate the border of a lesion in a captured image. \\ \hline

    Preconditions
    & None \\ \hline

    Acceptance Tests
    & \\ \hline

    Relations
    & UC-2 \\ \hline

    Comments
    &  \\ \hline

    \end{tabular}

    \caption{Functional Requirement 3}
    \label{fig:req_f_3}

\end{table}
\begin{table}[H]
    \begin{tabular}[t]{ | >{\bfseries}l | p{9.5cm} |}

    \hline
    ID
    &  REQ-F-4 \\ \hline

    Name
    & Browse Images and Results \\ \hline

    Description
    &  The system will let the user browse through a list of images. The list will display the status of the images. \\ \hline

    Preconditions
    & None \\ \hline

    Acceptance Tests
    & \\ \hline

    Relations
    &  \\ \hline

    Comments
    &  \\ \hline

    \end{tabular}

    \caption{Functional Requirement 4}
    \label{fig:req_f_4}

\end{table}
\begin{table}[H]
    \begin{tabular}[t]{ | >{\bfseries}l | p{9.5cm} |}

    \hline
    ID
    &  REQ-F-5 \\ \hline

    Name
    & Select Image To View in Detail \\ \hline

    Description
    &  The system will let the user select and view an image as well as the results of completed processes. \\ \hline

    Preconditions
    & None \\ \hline

    Acceptance Tests
    & \\ \hline

    Relations
    &  \\ \hline

    Comments
    &  \\ \hline

    \end{tabular}

    \caption{Functional Requirement 5}
    \label{fig:req_f_5}

\end{table}
\begin{table}[H]
    \begin{tabular}[t]{ | >{\bfseries}l | p{9.5cm} |}

    \hline
    ID
    &  REQ-F-6 \\ \hline

    Name
    & Border Confirmation \\ \hline

    Description
    & The system will let the user confirm that the border of a lesion hat been precisely calculated. \\ \hline

    Preconditions
    &  \\ \hline

    Acceptance Tests
    & \\ \hline

    Relations
    &  \\ \hline

    Comments
    &  \\ \hline

    \end{tabular}

    \caption{Functional Requirement 6}
    \label{fig:req_f_6}

\end{table}
\begin{table}[H]
    \begin{tabular}[t]{ | >{\bfseries}l | p{9.5cm} |}

    \hline
    ID
    &  REQ-F-7 \\ \hline

    Name
    & Feature Extraction \\ \hline

    Description
    & The system must be able to extract relavent features from the isolated lesion image. \\ \hline

    Preconditions
    &  \\ \hline

    Acceptance Tests
    & \\ \hline

    Relations
    &  \\ \hline

    Comments
    &  \\ \hline

    \end{tabular}

    \caption{Functional Requirement 7}
    \label{fig:req_f_7}

\end{table}
\begin{table}[H]
    \begin{tabular}[t]{ | >{\bfseries}l | p{9.5cm} |}

    \hline
    ID
    &  REQ-F-8 \\ \hline

    Name
    & Calculate Risk Assessment \\ \hline

    Description
    & The system must be able to calculate the risk assessment of an lesion being malignant. \\ \hline

    Preconditions
    &  \\ \hline

    Acceptance Tests
    & \\ \hline

    Relations
    &  \\ \hline

    Comments
    &  \\ \hline

    \end{tabular}

    \caption{Functional Requirement 8}
    \label{fig:req_f_8}

\end{table}
\begin{table}[H]
    \begin{tabular}[t]{ | >{\bfseries}l | p{9.5cm} |}

    \hline
    ID
    &  REQ-F-9 \\ \hline

    Name
    & Add/Save/Edit Metadata to Image and Results \\ \hline

    Description
    & The system will allow the user to add or edit metadata associated with the captured image. \\ \hline

    Preconditions
    &  \\ \hline

    Acceptance Tests
    & \\ \hline

    Relations
    &  \\ \hline

    Comments &
        The metadata associated might include the date when the image was captures as well as the location on the body where the lesion is located. The metadata will have the form of a text input field where the user can enter any text. It could be extended, if necessary, with structured data fields.

    \\ \hline

    \end{tabular}

    \caption{Functional Requirement 9}
    \label{fig:req_f_9}

\end{table}
\begin{table}[H]
    \begin{tabular}[t]{ | >{\bfseries}l | p{9.5cm} |}

    \hline
    ID
    &  REQ-F-10 \\ \hline

    Name
    & Save Image and Associated Data as Archive. \\ \hline

    Description
    & The system will allow the user to save images and associated data for reassessment and comparison in the future. \\ \hline

    Preconditions
    &  \\ \hline

    Acceptance Tests
    & \\ \hline

    Relations
    &  \\ \hline

    Comments &

        Storage on a mobile device is a limited resource. Depending on the development strategy chosen, Web App vs Native, the application might not have access to it's own data store. It would be useful to offload data to a cloud or web based storage location.

    \\ \hline

    \end{tabular}

    \caption{Functional Requirement 10}
    \label{fig:req_f_10}

\end{table}
\begin{table}[H]
    \begin{tabular}[t]{ | >{\bfseries}l | p{9.5cm} |}

    \hline
    ID
    &  REQ-F-11 \\ \hline

    Name
    & Browse Images and Results from Archive. \\ \hline

    Description
    & The sytem will allow the user to browse previously saved images and results from the archive. \\ \hline

    Preconditions
    &  \\ \hline

    Acceptance Tests
    & \\ \hline

    Relations
    &  \\ \hline

    Comments
    &  \\ \hline

    \end{tabular}

    \caption{Functional Requirement 11}
    \label{fig:req_f_11}

\end{table}





