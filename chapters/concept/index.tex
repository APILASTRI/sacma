\section{Use Cases}

\begin{table}[H]
    \begin{tabular}{ | >{\bfseries}l | p{9.5cm} |}
    \hline
    ID &  Short Identifier \\ \hline
    Name & Descriptive Title \\ \hline
    Description &  Use case in user story form\\ \hline
    Dependencies & dependencies \\ \hline
    Trigger & The event that starts the Use Case \\ \hline
    Preconditions & The system state which must be active before the Use Case can occur \\ \hline
    Normal Flow & Sequnce of events that occur during the use case. \\ \hline
    Alternate Flow & Alternative sequence of events that might occur. \\ \hline
    Results & triggers \\ \hline
    Comments & Other infos \\ \hline
    \end{tabular}

    \caption{Use Case Template}
    \label{fig:uc_template}
\end{table}

\begin{table}[H]
    \begin{tabular}{ | >{\bfseries}l | p{9.5cm} |}
    \hline
    ID
    &  UC-1 \\ \hline
    Name
    & Capture Image \\ \hline
    Description
    &  As a user I can capture an image of the skin lesion that I would like to have analysed. \\ \hline
    Dependencies
    & None \\ \hline
    Trigger
    & The user activated the application and selected the "image capture" navigation item. \\ \hline
    Preconditions
    & The system state which must be active before the Use Case can occur \\ \hline
    Normal Flow
    &

    \begin{description}[align=left]
    \item [1.]Point the camera at the skin lesion.
    \item [2.]Rotate and move the camera until the skin lesion is centered and optimally sized.
    \item [3.]The user touches the screen to capture the image.
    \item [4.]The user is notified ( beep ) that the image has been captured
    \item [5.]The user can repeat from the begining
    \end{description}

    \\ \hline
    Alternate Flow & None \\ \hline
    Results
    & The captured images are placed by the system in a processing queue to calculate the lesion's border. \\ \hline
    Comments
    & It is important that a user can capture the image with just one hand. If a lesion is located on a user's hand or arm, it's not possible to use two hands. \\ \hline
    \end{tabular}

    \caption{Use Case 1}
    \label{fig:uc_1}
\end{table}

\begin{table}[H]
    \begin{tabular}{ | >{\bfseries}l | p{9.5cm} |}
    \hline
    ID
    &  UC-2 \\ \hline
    Name
    & Confirm correct calculation of skin lesion's border \\ \hline
    Description
    &  As a user I want to confirm that the skin lesion's borders have been properly calculated. \\ \hline
    Dependencies
    & UC-1 \\ \hline
    Trigger
    & The user selected the "confirm border" navigation item. \\ \hline
    Preconditions
    & At least one image has been queued for border calculation. \\ \hline
    Normal Flow
    &
    \begin{description}[align=left]
    \item [1.]The user is presented with a list of images.
    \item [2.]The following step is repeated for each image in the list.
    \item [3.]The user confirms that the border of the lesion has been precisely calculated.
    \end{description}
    \\ \hline
    Alternate Flow
    &
    \begin{description}[align=left]
    \item [A1.] Border calculation for image has not completed.
    \item [A1.3] The user can refresh the image preview until the results of the border are visible.
    \item [A1.4] The user confirms that the border of the lesion has been precisely calculated.
    \end{description}
    \begin{description}[align=left]
    \item [A2] Border calculation is not precise or has failed.
    \item [A2.3] The user confirms that the border of the lesion has not been precisely calculated.
    \item [A2.4] The image is deleted.
    \end{description}
    \\ \hline
    Results
    & After positiv confirmation, images are placed by the system in the risk assessment calculation queue.
If no images can be positively confirmed, the user can recapture new images ( UC-1 ) \\ \hline
    Comments
    & It is important that a user can capture the image with just one hand. If a lesion is located on a user's hand or arm, it's not possible to use two hands. \\ \hline
    \end{tabular}

    \caption{Use Case 2}
    \label{fig:uc_2}
\end{table}

\section{Requirements}
\section{Prioritisation}
\section{Software Architecture}


https://en.wikipedia.org/wiki/Architectural\_pattern