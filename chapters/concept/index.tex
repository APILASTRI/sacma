In section 2.2.1 categories of medical smart phone apps were defined. Apps in the same categories might have a similar feature set and capabilities. An app in the \textit{Alert / First Response} category might have a database that stores important contacts in the user’s area and a user interface that allows the user to quickly alert a contact in an emergency situation. An app in the \textit{Learning / Educational / Reference} category might be modelled as a quiz app which presents the user with a question and several possible answers to choose from. Answers from several questions will can be evaluated and a score is presented to the user at the end of a session. The app would be implemented with a database of questions and weighted answers, as well as an interface that presents the questions to the user sequentially.

One of the goals of this project is to define requirements for an app that provides a risk assessment of a lesion being benign or malignant. A high level description of the primary functionality of the app is the following:

\begin{itemize}[label={}]
\item The smart phone app allows the user to capture and analyse an image of a skin lesion and provide a risk assessment to the user of the lesion being a malignant melanoma.

\end{itemize}

The user might also like to save the image and results in oder to be able to compare it with other assessments in the future, or to review the assessment with a dermatologist. For comparison it might be useful to save associated metadata, such as the date and location on the body of the lesion. For convenience it might be nice to send the assessment and image via email to a specialist for review.
Secondary functionality can be described as follows:

\begin{itemize}[label={}]
\item The app allows the user to save or archive the image and corresponding assessment for future comparison and review.

\item The user can add, edit, and save metadata associated with the image of the lesion.
\item The user can browse archived images, assessments, and associated metadata.
\item The user can send a set of images with associated assessment and metadata via email.

\end{itemize}

\section{Use Cases}

\begin{table}[H]
    \begin{tabular}{ | >{\bfseries}l | p{9.5cm} |}
    \hline
    ID &  Short Identifier \\ \hline
    Name & Descriptive Title \\ \hline
    Description &  Use case in user story form\\ \hline
    Dependencies & dependencies \\ \hline
    Trigger & The event that starts the Use Case \\ \hline
    Preconditions & The system state which must be active before the Use Case can occur \\ \hline
    Normal Flow & Sequnce of events that occur during the use case. \\ \hline
    Alternate Flow & Alternative sequence of events that might occur. \\ \hline
    Results & triggers \\ \hline
    Comments & Other infos \\ \hline
    \end{tabular}

    \caption{Use Case Template}
    \label{fig:uc_template}
\end{table}

\begin{table}[H]
    \begin{tabular}[t]{ | >{\bfseries}l | p{9.5cm} |}
    \hline
    ID
    &  UC-1 \\ \hline
    Name
    & Capture Image \\ \hline
    Description
    &  As a user I can capture an image of the skin lesion that I would like to have analysed. \\ \hline
    Dependencies
    & None \\ \hline
    Trigger
    & The user activated the application and selected the "image capture" navigation item. \\ \hline
    Preconditions
    & The system state which must be active before the Use Case can occur \\ \hline
    Normal Flow
    &

    \begin{description}[topsep=0cm,align=left]
        \item [1.]Point the camera at the skin lesion.
        \item [2.]Rotate and move the camera until the skin lesion is centered and optimally sized.
        \item [3.]The user touches the screen to capture the image.
        \item [4.]The user is notified ( beep ) that the image has been captured
        \item [5.]The user can repeat from the begining
    \end{description}

    \\ \hline
    Alternate Flow & None \\ \hline
    Results
    & The captured images are placed by the system in a processing queue to calculate the lesion's border. \\ \hline
    Comments
    & It is important that a user can capture the image with just one hand. If a lesion is located on a user's hand or arm, it's not possible to use two hands. \\ \hline
    \end{tabular}

    \caption{Use Case 1}
    \label{fig:uc_1}
\end{table}
\begin{table}[H]
    \begin{tabular}[t]{ | >{\bfseries}l | p{9.5cm} |} \hline

    ID &
        UC-2 \\ \hline
    Name &
        Confirm calculation of skin lesion’s border \\ \hline
    Priority &
        \\ \hline
    Description &
        As a user I want to confirm that the skin lesion’s borders have been properly calculated. \\ \hline
    Dependencies &
        UC-1 \\ \hline
    Trigger event &
        The user selected the ”confirm border” navigation item. \\ \hline
    Preconditions &
        At least one image has been captured and border calculation could complete. \\ \hline
    Postconditions &
        \\ \hline
    Result &
        After positiv confirmation, the system initiates the risk assessment calculation for the image. If the image can not be positively confirmed, the user can recapture new images ( UC-1 ) \\ \hline
    Main scenario
    &

        \begin{description}[align=left]
            \item [1.]The user is presented with a list of images.
            \item [2.]The following step is repeated for each image in the list.
            \item [3.]The user confirms that the border of the lesion has been precisely calculated.
        \end{description}

    \\ \hline

    Alternate scenarios &
        \begin{description}[align=left]
            \item [A1.] Border calculation for image has not completed.
            \item [A1.3] The user can refresh the image preview until the results of the border are visible.
            \item [A1.4] The user confirms that the border of the lesion has been precisely calculated.
        \end{description}
        \begin{description}[align=left]
            \item [A2] Border calculation is not precise or has failed.
            \item [A2.3] The user confirms that the border of the lesion has not been precisely calculated.
            \item [A2.4] The image is deleted.
        \end{description}
    \\hline

    Comments &
        \\ \hline

    \end{tabular}
  \caption{Use Case 2}
  \label{fig:uc_2}
\end{table}
\begin{table}[H]
    \begin{tabular}[t]{ | >{\bfseries}l | p{9.5cm} |} \hline

    ID &
        UC-3 \\
    \hline

    Name &
        View risk assessment \\
    \hline

    Priority &
        \\
    \hline

    Description &
        As a user I want to view the results of the risk assessment calculation. \\
    \hline

    Dependencies &
        UC-2 \\
    \hline

    Trigger event &
        The user selected the "risk assessment results" navigation item. \\
    \hline

    Preconditions &
        The system was able to complete the risk assessment calculation. \\
    \hline

    Postconditions &
        \\
    \hline

    Result &
         \\
     \hline

    Main scenario
    &

    \begin{description}[topsep=0cm,align=left]
        \item [1.]The user is presented with a list of images.
        \item [2.]The following step is repeated for each image in the list.
        \item [3.]The user can select and view and image and corresponding results in detail.

    \end{description}

    \\ \hline

    Alternate scenarios &
        \begin{description}[align=left]
            \item [A1.] Risk assessment calculation for image has not completed.
            \item [A1.3] The user can refresh the results details until the results of a risk assessment is available.
        \end{description}
    \\ \hline

    Comments &
        \\
    \hline

    \end{tabular}
  \caption{Use Case 3}
  \label{fig:uc_3}
\end{table}
\begin{table}[H]
    \begin{tabular}{ | >{\bfseries}l | p{9.5cm} |}
    \hline
    ID
    &  UC-4 \\ \hline
    Name
    & Save image and results \\ \hline
    Description
    &  As a user I want to save the image and the results of the risk assessment. \\ \hline
    Dependencies
    & UC-3 \\ \hline
    Trigger
    & The user selects the "save" navigation item. \\ \hline
    Preconditions
    & The system was able to complete the risk assessment calculation. \\ \hline
    Normal Flow
    &
    \begin{description}[align=left]
    \item [1.]The user is presented with a list of images.
    \item [2.]The can select one or more images with the corresponding risk assessment results.
    \item [3.]The user can add a title and comment.
    \item [4.]The user is presented with a confirmation that the package has been saved.
    \end{description}
    \\ \hline
    Alternate Flow
    &

    \\ \hline
    Results
    &
    A package of images and corresponding risk assessment data is stored in the archive with some metadata (title and comment).
    \\ \hline
    Comments
    &  \\ \hline
    \end{tabular}

    \caption{Use Case 4}
    \label{fig:uc_4}
\end{table}
\begin{table}[H]
    \begin{tabular}{ | >{\bfseries}l | p{9.5cm} |}
    \hline
    ID
    &  UC-5 \\ \hline
    Name
    & View archived images and data. \\ \hline
    Description
    &  As a user I want to view previously saved images and the corresponding risk assessment data. \\ \hline
    Dependencies
    & UC-4 \\ \hline
    Trigger
    & The user selects the "archive" navigation item. \\ \hline
    Preconditions
    & The system was able to complete the risk assessment calculation. \\ \hline
    Normal Flow
    &
    \begin{description}[align=left]
    \item [1.]The user is presented with a list of archived packages.
    \item [2.]The user can select an item in the list
    \item [3.]The user is presented with the images, corresponding risk assessment data and metadata for the selected item.
    \end{description}
    \\ \hline
    Alternate Flow
    &

    \\ \hline
    Results
    &
    A package of images and corresponding risk assessment data is stored in the archive with some metadata (title and comment).
    \\ \hline
    Comments
    &  \\ \hline
    \end{tabular}

    \caption{Use Case 5}
    \label{fig:uc_5}
\end{table}
\begin{table}[H]
    \begin{tabular}{ | >{\bfseries}l | p{9.5cm} |}
    \hline
    ID
    &  UC-6 \\ \hline
    Name
    & Send results to dermatologist. \\ \hline
    Description
    &  As a user I want to send a previously saved image, feature extraction data and assessment results to a dermatologist. \\ \hline
    Dependencies
    & UC-4 \\ \hline
    Trigger
    & The user selects the "archive" navigation item. \\ \hline
    Preconditions
    & The system was able to complete the risk assessment calculation. \\ \hline
    Normal Flow
    &
    \begin{description}[align=left]
    \item [1.]The user is presented with a list of archived packages.
    \item [2.]The user can select an item in the list
    \item [3.]The user can send the item as an email attachment.
    \end{description}
    \\ \hline
    Alternate Flow
    &

    \\ \hline
    Results
    &
    \\ \hline
    Comments
    &  \\ \hline
    \end{tabular}

    \caption{Use Case 6}
    \label{fig:uc_6}
\end{table}

\section{User Interface}
\begin{figure}[H]
\centering
    \includegraphics[height=10cm,keepaspectratio]{assets/GUI/image_capture.pdf}
    \caption{Image Capture View}
    \label{fig:image_capture}
\end{figure}

\begin{figure}[H]
    \centering
    \includegraphics[height=10cm,keepaspectratio]{assets/GUI/image_list_view.pdf}
    \caption{Image List View}
    \label{fig:image_list_view}
\end{figure}

\begin{figure}[H]
    \centering
    \includegraphics[height=10cm,keepaspectratio]{assets/GUI/border_confirm.pdf}
    \caption{Border Confirm View}
    \label{fig:border_confirm_view}
\end{figure}

\begin{figure}[H]
    \centering
    \includegraphics[height=10cm,keepaspectratio]{assets/GUI/image_detail_view.pdf}
    \caption{Image Detail View}
    \label{fig:image_detail_view}
\end{figure}

\begin{figure}[H]
    \centering
    \includegraphics[height=10cm,keepaspectratio]{assets/GUI/archive_view.pdf}
    \caption{Archive View}
    \label{fig:archive_view}
\end{figure}

\begin{figure}[H]
    \centering
    \includegraphics[height=10cm,keepaspectratio]{assets/GUI/archive_detail_view.pdf}
    \caption{Archive Detail View}
    \label{fig:archive_detail}
\end{figure}

\section{Requirements}

\section{Prioritized Requirements}

\begin{longtable}{ | l | l | l | l | l | l | l |}
\hline
ID & Priority & Name & Implemented  \\ \hline

REQ-F-1 & - & Preview Camera Input & -  \\ \hline
REQ-F-2 & - & Capture Camera Input & -  \\ \hline
REQ-F-3 & - & Border Extraction & -  \\ \hline
REQ-F-4 & - & Browse Images and Results & -  \\ \hline
REQ-F-5 & - & Select Image To View in Detail & -  \\ \hline
REQ-F-6 & - & Border Confirmation & -  \\ \hline
REQ-F-7 & - & Feature Extraction & -  \\ \hline
REQ-F-8 & - & Feature Extraction & -  \\ \hline


\caption{Prioritzed Requirement List}
\label{fig:prio_req}
\end{longtable}

\begin{table}[H]
    \begin{tabular}[t]{ | >{\bfseries}l | p{9.5cm} |}

    \hline
    ID
    &  REQ-F-1 \\ \hline

    Name
    & Preview Camera Input \\ \hline

    Description
    &  The system will let the user view a preview of the camera's input in realtime \\ \hline

    Preconditions
    & None \\ \hline

    Acceptance Tests
    & \\ \hline

    Relations
    & UC-1 \\ \hline

    Comments
    &  \\ \hline

    \end{tabular}

    \caption{Functional Requirement 1}
    \label{fig:req_f_1}

\end{table}
\begin{table}[H]
    \begin{tabular}[t]{ | >{\bfseries}l | p{9.5cm} |}

    \hline
    ID
    &  REQ-F-1 \\ \hline

    Name
    & Capture Camera Input \\ \hline

    Description
    &  The system will let the user capture an image from the camera's input. \\ \hline

    Preconditions
    & None \\ \hline

    Acceptance Tests
    & \\ \hline

    Relations
    & UC-1 \\ \hline

    Comments
    &  \\ \hline

    \end{tabular}

    \caption{Functional Requirement 1}
    \label{fig:req_f_1}

\end{table}
\begin{table}[H]
    \begin{tabular}[t]{ | >{\bfseries}l | p{9.5cm} |}

    \hline
    ID
    &  REQ-F-3 \\ \hline

    Name
    & Border Extraction \\ \hline

    Description
    &  The system must be able to calculate the border of a lesion in a captured image. \\ \hline

    Preconditions
    & None \\ \hline

    Acceptance Tests
    & \\ \hline

    Relations
    & UC-2 \\ \hline

    Comments
    &  \\ \hline

    \end{tabular}

    \caption{Functional Requirement 3}
    \label{fig:req_f_3}

\end{table}
\begin{table}[H]
    \begin{tabular}[t]{ | >{\bfseries}l | p{9.5cm} |}

    \hline
    ID
    &  REQ-F-4 \\ \hline

    Name
    & Browse Images and Results \\ \hline

    Description
    &  The system will let the user browse through a list of images. The list will display the status of the images. \\ \hline

    Preconditions
    & None \\ \hline

    Acceptance Tests
    & \\ \hline

    Relations
    &  \\ \hline

    Comments
    &  \\ \hline

    \end{tabular}

    \caption{Functional Requirement 4}
    \label{fig:req_f_4}

\end{table}
\begin{table}[H]
    \begin{tabular}[t]{ | >{\bfseries}l | p{9.5cm} |}

    \hline
    ID
    &  REQ-F-5 \\ \hline

    Name
    & Select Image To View in Detail \\ \hline

    Description
    &  The system will let the user select and view an image as well as the results of completed processes. \\ \hline

    Preconditions
    & None \\ \hline

    Acceptance Tests
    & \\ \hline

    Relations
    &  \\ \hline

    Comments
    &  \\ \hline

    \end{tabular}

    \caption{Functional Requirement 5}
    \label{fig:req_f_5}

\end{table}
\begin{table}[H]
    \begin{tabular}[t]{ | >{\bfseries}l | p{9.5cm} |}

    \hline
    ID
    &  REQ-F-6 \\ \hline

    Name
    & Border Confirmation \\ \hline

    Description
    & The system will let the user confirm that the border of a lesion hat been precisely calculated. \\ \hline

    Preconditions
    &  \\ \hline

    Acceptance Tests
    & \\ \hline

    Relations
    &  \\ \hline

    Comments
    &  \\ \hline

    \end{tabular}

    \caption{Functional Requirement 6}
    \label{fig:req_f_6}

\end{table}
\begin{table}[H]
    \begin{tabular}[t]{ | >{\bfseries}l | p{9.5cm} |}

    \hline
    ID
    &  REQ-F-7 \\ \hline

    Name
    & Feature Extraction \\ \hline

    Description
    & The system must be able to extract relavent features from the isolated lesion image. \\ \hline

    Preconditions
    &  \\ \hline

    Acceptance Tests
    & \\ \hline

    Relations
    &  \\ \hline

    Comments
    &  \\ \hline

    \end{tabular}

    \caption{Functional Requirement 7}
    \label{fig:req_f_7}

\end{table}
\begin{table}[H]
    \begin{tabular}[t]{ | >{\bfseries}l | p{9.5cm} |}

    \hline
    ID
    &  REQ-F-8 \\ \hline

    Name
    & Add To Process Queue \\ \hline

    Description
    & The system must be able to add an image process job to the process queue \\ \hline

    Preconditions
    &  \\ \hline

    Acceptance Tests
    & \\ \hline

    Relations
    &  \\ \hline

    Comments
    &  \\ \hline

    \end{tabular}

    \caption{Functional Requirement 8}
    \label{fig:req_f_8}

\end{table}
\begin{table}[H]
    \begin{tabular}[t]{ | >{\bfseries}l | p{9.5cm} |}

    \hline
    ID
    &  REQ-F-9 \\ \hline

    Name
    & Asign Image to Process \\ \hline

    Description
    & The system must be able to asign an image process to an image process job \\ \hline

    Preconditions
    &  \\ \hline

    Acceptance Tests
    & \\ \hline

    Relations
    &  \\ \hline

    Comments
    &  \\ \hline

    \end{tabular}

    \caption{Functional Requirement 9}
    \label{fig:req_f_9}

\end{table}
\begin{table}[H]
    \begin{tabular}[t]{ | >{\bfseries}l | p{9.5cm} |}

    \hline
    ID
    &  REQ-F-10 \\ \hline

    Name
    & Remove From Queue \\ \hline

    Description
    & When an image process job in complete it will be removed from the process queue. \\ \hline

    Preconditions
    &  \\ \hline

    Acceptance Tests
    & \\ \hline

    Relations
    &  \\ \hline

    Comments
    &  \\ \hline

    \end{tabular}

    \caption{Functional Requirement 10}
    \label{fig:req_f_10}

\end{table}
\begin{table}[H]
    \begin{tabular}[t]{ | >{\bfseries}l | p{9.5cm} |}

    \hline
    ID
    &  REQ-F-11 \\ \hline

    Name
    & Browse Images and Results from Archive. \\ \hline

    Description
    & The sytem will allow the user to browse previously saved images and results from the archive. \\ \hline

    Preconditions
    &  \\ \hline

    Acceptance Tests
    & \\ \hline

    Relations
    &  \\ \hline

    Comments
    &  \\ \hline

    \end{tabular}

    \caption{Functional Requirement 11}
    \label{fig:req_f_11}

\end{table}

\section{Prioritisation}
\section{Software Architecture}


https://en.wikipedia.org/wiki/Architectural\_pattern