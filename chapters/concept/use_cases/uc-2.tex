\begin{table}[H]
    \begin{tabular}{ | >{\bfseries}l | p{9.5cm} |}
    \hline
    ID
    &  UC-2 \\ \hline
    Name
    & Confirm correct calculation of skin lesion's border \\ \hline
    Description
    &  As a user I want to confirm that the skin lesion's borders have been properly calculated. \\ \hline
    Dependencies
    & UC-1 \\ \hline
    Trigger
    & The user selected the "confirm border" navigation item. \\ \hline
    Preconditions
    & At least one image has been queued for border calculation. \\ \hline
    Normal Flow
    &
    \begin{description}[align=left]
    \item [1.]The user is presented with a list of images.
    \item [2.]The following step is repeated for each image in the list.
    \item [3.]The user confirms that the border of the lesion has been precisely calculated.
    \end{description}
    \\ \hline
    Alternate Flow
    &
    \begin{description}[align=left]
    \item [A1.] Border calculation for image has not completed.
    \item [A1.3] The user can refresh the image preview until the results of the border are visible.
    \item [A1.4] The user confirms that the border of the lesion has been precisely calculated.
    \end{description}
    \begin{description}[align=left]
    \item [A2] Border calculation is not precise or has failed.
    \item [A2.3] The user confirms that the border of the lesion has not been precisely calculated.
    \item [A2.4] The image is deleted.
    \end{description}
    \\ \hline
    Results
    & After positiv confirmation, images are placed by the system in the risk assessment calculation queue.
If no images can be positively confirmed, the user can recapture new images ( UC-1 ) \\ \hline
    Comments
    & It is important that a user can capture the image with just one hand. If a lesion is located on a user's hand or arm, it's not possible to use two hands. \\ \hline
    \end{tabular}

    \caption{Use Case 2}
    \label{fig:uc_2}
\end{table}