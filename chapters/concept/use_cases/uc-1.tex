\begin{table}[H]
    \begin{tabular}[t]{ | >{\bfseries}l | p{9.5cm} |}
    \hline
    ID
    &  UC-1 \\ \hline
    Name
    & Capture Image \\ \hline
    Description
    &  As a user I can capture an image of the skin lesion that I would like to have analysed. \\ \hline
    Dependencies
    & None \\ \hline
    Trigger
    & The user activated the application and selected the "image capture" navigation item. \\ \hline
    Preconditions
    & The system state which must be active before the Use Case can occur \\ \hline
    Normal Flow
    &

    \begin{description}[topsep=0cm,align=left]
        \item [1.]Point the camera at the skin lesion.
        \item [2.]Rotate and move the camera until the skin lesion is centered and optimally sized.
        \item [3.]The user touches the screen to capture the image.
        \item [4.]The user is notified ( beep ) that the image has been captured
        \item [5.]The user can repeat from the begining
    \end{description}

    \\ \hline
    Alternate Flow & None \\ \hline
    Results
    & The captured images are placed by the system in a processing queue to calculate the lesion's border. \\ \hline
    Comments
    & It is important that a user can capture the image with just one hand. If a lesion is located on a user's hand or arm, it's not possible to use two hands. \\ \hline
    \end{tabular}

    \caption{Use Case 1}
    \label{fig:uc_1}
\end{table}