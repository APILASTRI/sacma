This project will employ a Single-Criterion Ad-hoc prioritisation classification as opposed to an analytical approach such as the Wiegers prioritisation matrix. The rationale behind this choice is that as a single person development project the determination of weights for benefit, detriment, cost, and risk is an ad-hoc exercise. In a larger project with multiple developers and active stakeholders the time and effort required for a prioritisation assessment according to Wiegers would be advisable.

The following are the prioritisation classes as defined in \cite{9781937538774}:

\begin{itemize}[label={}]

\item \textbf{Mandatory}: A mandatory requirement is a requirement that must be implemented at all costs or else the success of the system is threatened.
\item \textbf{Optional}: An optional requirement is a requirement that does not necessarily need to be implemented. Neglecting a few requirements of this class does not threaten the success of the system.
\item \textbf{Nice-to-have}: Nice-to-have requirements are requirements that do not influence the system’s success if they are not implemented.

\end{itemize}